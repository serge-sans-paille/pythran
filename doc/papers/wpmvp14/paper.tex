% submit to https://sites.google.com/site/wpmvp2014/home
\documentclass[preprint]{sigplanconf}

% The following \documentclass options may be useful:

% preprint      Remove this option only once the paper is in final form.
% 10pt          To set in 10-point type instead of 9-point.
% 11pt          To set in 11-point type instead of 9-point.
% authoryear    To obtain author/year citation style instead of numeric.

\usepackage{amsmath}
\usepackage{listings}
\usepackage{hyperref}


\begin{document}

\special{papersize=8.5in,11in}
\setlength{\pdfpageheight}{\paperheight}
\setlength{\pdfpagewidth}{\paperwidth}

%\conferenceinfo{CONF 'yy}{Month d--d, 20yy, City, ST, Country} 
%\copyrightyear{20yy} 
%\copyrightdata{978-1-nnnn-nnnn-n/yy/mm} 
%\doi{nnnnnnn.nnnnnnn}

\title{Vectorizing Python Constructs Using Pythran and Boost SIMD}

\authorinfo{Serge Guelton}
           {QuarksLab, T{\'e}l{\'e}com Bretagne}
           {sguelton@quarkslab.com}
\authorinfo{Jo{\"e}l Falcou}
           {MetaScale}
           {joel.falcou@metascale.org}

\maketitle

\begin{abstract}

    The Python language is highly dynamic, most notably due to late binding. As
    a consequence, program run using Python typically run an order of magnitude
    slower than their C counterpart. It is also a high level languages whose
    semantic can be made more static without much change from a user point of
    view in the case of mathematical applications. In that case, the language
    provides several vectorization opportunities that are studied in this
    paper, and evaluated in the context of Pythran, an ahead-of-time compiler
    that turns Python module into C++ meta-programs.

\end{abstract}

%\category{CR-number}{subcategory}{third-level}

% general terms are not compulsory anymore, 
% you may leave them out
%\terms
%term1, term2

\keywords
Vectorization, Meta-Programming, Python, C++


%%
%%
\section{Python, Unboxing and Pythran}

The Python language has grown in audience for the past ten years, even reaching
the world of scientific computations thanks to the \texttt{numpy} module, a
module that provides a MATLAB-like API. As a consequence, more and more code is
being written either in pure Python, generally to prototype an application, or
as a Python and native code mix when performance matters. However, Python
trades performance for dynamicity and does not particularly shines in terms of
performance.

For instance, the natural solution to Project Euler sixth problem ``Find the
difference between the sum of the squares of the first one hundred natural
numbers and the square of the sum`` can be coded as in
Listing~\ref{lst:euler06-py} in Python and as in Listing~\ref{lst:euler06-c} in
C. A comparison of the performance of the Python function and the C function
called through the \texttt{ctypes} module shows that the C version runs more
than $\times250$ faster than the Python version. Enabling compiler
auto-vectorization makes the C version $\times400$ faster than the Python
version.~\footnote{To perform more reliable time measurements, \texttt{n} has
    been increased to 100001 instead of 101. The script used to perform the
    measures as well as all other experimental data are available at
\url{https://github.com/serge-sans-paille/pythran/tree/wpmvp14/doc/papers/wpmvp14/experiments}.}

\lstinputlisting[language=python, label={lst:euler06-py}, caption={Solution to the Project Euler Sixth Problem in Python}]{experiments/euler06.py}

\lstinputlisting[language=c, label={lst:euler06-c}, caption={Solution to the Project Euler Sixth Problem in C}]{experiments/euler06.c}

\subsection{The Unboxing Problem}

\subsection{The Pythran Compiler}

%%
%%
\section{Changing Python Semantic for Vectorization}

\subsection{Intrinsics}

\subsection{List Comprehension}

\subsection{The Numpy Module}

%%
%%
\section{Vectorization of a Meta-Program}

\subsection{Boost SIMD}

\subsection{Load/Store Protocol}

%%
%%
\section{Validation of the Approach using the Pythran Compiler}

\acks

Acknowledgments, if needed.

% We recommend abbrvnat bibliography style.
\cite{*}

\bibliographystyle{abbrvnat}
\bibliography{biblio}


\end{document}
